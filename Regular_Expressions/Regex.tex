\documentclass[11pt]{article}

    \usepackage[breakable]{tcolorbox}
    \usepackage{parskip} % Stop auto-indenting (to mimic markdown behaviour)
    
    \usepackage{iftex}
    \ifPDFTeX
    	\usepackage[T1]{fontenc}
    	\usepackage{mathpazo}
    \else
    	\usepackage{fontspec}
    \fi

    % Basic figure setup, for now with no caption control since it's done
    % automatically by Pandoc (which extracts ![](path) syntax from Markdown).
    \usepackage{graphicx}
    % Maintain compatibility with old templates. Remove in nbconvert 6.0
    \let\Oldincludegraphics\includegraphics
    % Ensure that by default, figures have no caption (until we provide a
    % proper Figure object with a Caption API and a way to capture that
    % in the conversion process - todo).
    \usepackage{caption}
    \DeclareCaptionFormat{nocaption}{}
    \captionsetup{format=nocaption,aboveskip=0pt,belowskip=0pt}

    \usepackage[Export]{adjustbox} % Used to constrain images to a maximum size
    \adjustboxset{max size={0.9\linewidth}{0.9\paperheight}}
    \usepackage{float}
    \floatplacement{figure}{H} % forces figures to be placed at the correct location
    \usepackage{xcolor} % Allow colors to be defined
    \usepackage{enumerate} % Needed for markdown enumerations to work
    \usepackage{geometry} % Used to adjust the document margins
    \usepackage{amsmath} % Equations
    \usepackage{amssymb} % Equations
    \usepackage{textcomp} % defines textquotesingle
    % Hack from http://tex.stackexchange.com/a/47451/13684:
    \AtBeginDocument{%
        \def\PYZsq{\textquotesingle}% Upright quotes in Pygmentized code
    }
    \usepackage{upquote} % Upright quotes for verbatim code
    \usepackage{eurosym} % defines \euro
    \usepackage[mathletters]{ucs} % Extended unicode (utf-8) support
    \usepackage{fancyvrb} % verbatim replacement that allows latex
    \usepackage{grffile} % extends the file name processing of package graphics 
                         % to support a larger range
    \makeatletter % fix for grffile with XeLaTeX
    \def\Gread@@xetex#1{%
      \IfFileExists{"\Gin@base".bb}%
      {\Gread@eps{\Gin@base.bb}}%
      {\Gread@@xetex@aux#1}%
    }
    \makeatother

    % The hyperref package gives us a pdf with properly built
    % internal navigation ('pdf bookmarks' for the table of contents,
    % internal cross-reference links, web links for URLs, etc.)
    \usepackage{hyperref}
    % The default LaTeX title has an obnoxious amount of whitespace. By default,
    % titling removes some of it. It also provides customization options.
    \usepackage{titling}
    \usepackage{longtable} % longtable support required by pandoc >1.10
    \usepackage{booktabs}  % table support for pandoc > 1.12.2
    \usepackage[inline]{enumitem} % IRkernel/repr support (it uses the enumerate* environment)
    \usepackage[normalem]{ulem} % ulem is needed to support strikethroughs (\sout)
                                % normalem makes italics be italics, not underlines
    \usepackage{mathrsfs}
    

    
    % Colors for the hyperref package
    \definecolor{urlcolor}{rgb}{0,.145,.698}
    \definecolor{linkcolor}{rgb}{.71,0.21,0.01}
    \definecolor{citecolor}{rgb}{.12,.54,.11}

    % ANSI colors
    \definecolor{ansi-black}{HTML}{3E424D}
    \definecolor{ansi-black-intense}{HTML}{282C36}
    \definecolor{ansi-red}{HTML}{E75C58}
    \definecolor{ansi-red-intense}{HTML}{B22B31}
    \definecolor{ansi-green}{HTML}{00A250}
    \definecolor{ansi-green-intense}{HTML}{007427}
    \definecolor{ansi-yellow}{HTML}{DDB62B}
    \definecolor{ansi-yellow-intense}{HTML}{B27D12}
    \definecolor{ansi-blue}{HTML}{208FFB}
    \definecolor{ansi-blue-intense}{HTML}{0065CA}
    \definecolor{ansi-magenta}{HTML}{D160C4}
    \definecolor{ansi-magenta-intense}{HTML}{A03196}
    \definecolor{ansi-cyan}{HTML}{60C6C8}
    \definecolor{ansi-cyan-intense}{HTML}{258F8F}
    \definecolor{ansi-white}{HTML}{C5C1B4}
    \definecolor{ansi-white-intense}{HTML}{A1A6B2}
    \definecolor{ansi-default-inverse-fg}{HTML}{FFFFFF}
    \definecolor{ansi-default-inverse-bg}{HTML}{000000}

    % commands and environments needed by pandoc snippets
    % extracted from the output of `pandoc -s`
    \providecommand{\tightlist}{%
      \setlength{\itemsep}{0pt}\setlength{\parskip}{0pt}}
    \DefineVerbatimEnvironment{Highlighting}{Verbatim}{commandchars=\\\{\}}
    % Add ',fontsize=\small' for more characters per line
    \newenvironment{Shaded}{}{}
    \newcommand{\KeywordTok}[1]{\textcolor[rgb]{0.00,0.44,0.13}{\textbf{{#1}}}}
    \newcommand{\DataTypeTok}[1]{\textcolor[rgb]{0.56,0.13,0.00}{{#1}}}
    \newcommand{\DecValTok}[1]{\textcolor[rgb]{0.25,0.63,0.44}{{#1}}}
    \newcommand{\BaseNTok}[1]{\textcolor[rgb]{0.25,0.63,0.44}{{#1}}}
    \newcommand{\FloatTok}[1]{\textcolor[rgb]{0.25,0.63,0.44}{{#1}}}
    \newcommand{\CharTok}[1]{\textcolor[rgb]{0.25,0.44,0.63}{{#1}}}
    \newcommand{\StringTok}[1]{\textcolor[rgb]{0.25,0.44,0.63}{{#1}}}
    \newcommand{\CommentTok}[1]{\textcolor[rgb]{0.38,0.63,0.69}{\textit{{#1}}}}
    \newcommand{\OtherTok}[1]{\textcolor[rgb]{0.00,0.44,0.13}{{#1}}}
    \newcommand{\AlertTok}[1]{\textcolor[rgb]{1.00,0.00,0.00}{\textbf{{#1}}}}
    \newcommand{\FunctionTok}[1]{\textcolor[rgb]{0.02,0.16,0.49}{{#1}}}
    \newcommand{\RegionMarkerTok}[1]{{#1}}
    \newcommand{\ErrorTok}[1]{\textcolor[rgb]{1.00,0.00,0.00}{\textbf{{#1}}}}
    \newcommand{\NormalTok}[1]{{#1}}
    
    % Additional commands for more recent versions of Pandoc
    \newcommand{\ConstantTok}[1]{\textcolor[rgb]{0.53,0.00,0.00}{{#1}}}
    \newcommand{\SpecialCharTok}[1]{\textcolor[rgb]{0.25,0.44,0.63}{{#1}}}
    \newcommand{\VerbatimStringTok}[1]{\textcolor[rgb]{0.25,0.44,0.63}{{#1}}}
    \newcommand{\SpecialStringTok}[1]{\textcolor[rgb]{0.73,0.40,0.53}{{#1}}}
    \newcommand{\ImportTok}[1]{{#1}}
    \newcommand{\DocumentationTok}[1]{\textcolor[rgb]{0.73,0.13,0.13}{\textit{{#1}}}}
    \newcommand{\AnnotationTok}[1]{\textcolor[rgb]{0.38,0.63,0.69}{\textbf{\textit{{#1}}}}}
    \newcommand{\CommentVarTok}[1]{\textcolor[rgb]{0.38,0.63,0.69}{\textbf{\textit{{#1}}}}}
    \newcommand{\VariableTok}[1]{\textcolor[rgb]{0.10,0.09,0.49}{{#1}}}
    \newcommand{\ControlFlowTok}[1]{\textcolor[rgb]{0.00,0.44,0.13}{\textbf{{#1}}}}
    \newcommand{\OperatorTok}[1]{\textcolor[rgb]{0.40,0.40,0.40}{{#1}}}
    \newcommand{\BuiltInTok}[1]{{#1}}
    \newcommand{\ExtensionTok}[1]{{#1}}
    \newcommand{\PreprocessorTok}[1]{\textcolor[rgb]{0.74,0.48,0.00}{{#1}}}
    \newcommand{\AttributeTok}[1]{\textcolor[rgb]{0.49,0.56,0.16}{{#1}}}
    \newcommand{\InformationTok}[1]{\textcolor[rgb]{0.38,0.63,0.69}{\textbf{\textit{{#1}}}}}
    \newcommand{\WarningTok}[1]{\textcolor[rgb]{0.38,0.63,0.69}{\textbf{\textit{{#1}}}}}
    
    
    % Define a nice break command that doesn't care if a line doesn't already
    % exist.
    \def\br{\hspace*{\fill} \\* }
    % Math Jax compatibility definitions
    \def\gt{>}
    \def\lt{<}
    \let\Oldtex\TeX
    \let\Oldlatex\LaTeX
    \renewcommand{\TeX}{\textrm{\Oldtex}}
    \renewcommand{\LaTeX}{\textrm{\Oldlatex}}
    % Document parameters
    % Document title
    \title{Regex}
    
    
    
    
    
% Pygments definitions
\makeatletter
\def\PY@reset{\let\PY@it=\relax \let\PY@bf=\relax%
    \let\PY@ul=\relax \let\PY@tc=\relax%
    \let\PY@bc=\relax \let\PY@ff=\relax}
\def\PY@tok#1{\csname PY@tok@#1\endcsname}
\def\PY@toks#1+{\ifx\relax#1\empty\else%
    \PY@tok{#1}\expandafter\PY@toks\fi}
\def\PY@do#1{\PY@bc{\PY@tc{\PY@ul{%
    \PY@it{\PY@bf{\PY@ff{#1}}}}}}}
\def\PY#1#2{\PY@reset\PY@toks#1+\relax+\PY@do{#2}}

\expandafter\def\csname PY@tok@w\endcsname{\def\PY@tc##1{\textcolor[rgb]{0.73,0.73,0.73}{##1}}}
\expandafter\def\csname PY@tok@c\endcsname{\let\PY@it=\textit\def\PY@tc##1{\textcolor[rgb]{0.25,0.50,0.50}{##1}}}
\expandafter\def\csname PY@tok@cp\endcsname{\def\PY@tc##1{\textcolor[rgb]{0.74,0.48,0.00}{##1}}}
\expandafter\def\csname PY@tok@k\endcsname{\let\PY@bf=\textbf\def\PY@tc##1{\textcolor[rgb]{0.00,0.50,0.00}{##1}}}
\expandafter\def\csname PY@tok@kp\endcsname{\def\PY@tc##1{\textcolor[rgb]{0.00,0.50,0.00}{##1}}}
\expandafter\def\csname PY@tok@kt\endcsname{\def\PY@tc##1{\textcolor[rgb]{0.69,0.00,0.25}{##1}}}
\expandafter\def\csname PY@tok@o\endcsname{\def\PY@tc##1{\textcolor[rgb]{0.40,0.40,0.40}{##1}}}
\expandafter\def\csname PY@tok@ow\endcsname{\let\PY@bf=\textbf\def\PY@tc##1{\textcolor[rgb]{0.67,0.13,1.00}{##1}}}
\expandafter\def\csname PY@tok@nb\endcsname{\def\PY@tc##1{\textcolor[rgb]{0.00,0.50,0.00}{##1}}}
\expandafter\def\csname PY@tok@nf\endcsname{\def\PY@tc##1{\textcolor[rgb]{0.00,0.00,1.00}{##1}}}
\expandafter\def\csname PY@tok@nc\endcsname{\let\PY@bf=\textbf\def\PY@tc##1{\textcolor[rgb]{0.00,0.00,1.00}{##1}}}
\expandafter\def\csname PY@tok@nn\endcsname{\let\PY@bf=\textbf\def\PY@tc##1{\textcolor[rgb]{0.00,0.00,1.00}{##1}}}
\expandafter\def\csname PY@tok@ne\endcsname{\let\PY@bf=\textbf\def\PY@tc##1{\textcolor[rgb]{0.82,0.25,0.23}{##1}}}
\expandafter\def\csname PY@tok@nv\endcsname{\def\PY@tc##1{\textcolor[rgb]{0.10,0.09,0.49}{##1}}}
\expandafter\def\csname PY@tok@no\endcsname{\def\PY@tc##1{\textcolor[rgb]{0.53,0.00,0.00}{##1}}}
\expandafter\def\csname PY@tok@nl\endcsname{\def\PY@tc##1{\textcolor[rgb]{0.63,0.63,0.00}{##1}}}
\expandafter\def\csname PY@tok@ni\endcsname{\let\PY@bf=\textbf\def\PY@tc##1{\textcolor[rgb]{0.60,0.60,0.60}{##1}}}
\expandafter\def\csname PY@tok@na\endcsname{\def\PY@tc##1{\textcolor[rgb]{0.49,0.56,0.16}{##1}}}
\expandafter\def\csname PY@tok@nt\endcsname{\let\PY@bf=\textbf\def\PY@tc##1{\textcolor[rgb]{0.00,0.50,0.00}{##1}}}
\expandafter\def\csname PY@tok@nd\endcsname{\def\PY@tc##1{\textcolor[rgb]{0.67,0.13,1.00}{##1}}}
\expandafter\def\csname PY@tok@s\endcsname{\def\PY@tc##1{\textcolor[rgb]{0.73,0.13,0.13}{##1}}}
\expandafter\def\csname PY@tok@sd\endcsname{\let\PY@it=\textit\def\PY@tc##1{\textcolor[rgb]{0.73,0.13,0.13}{##1}}}
\expandafter\def\csname PY@tok@si\endcsname{\let\PY@bf=\textbf\def\PY@tc##1{\textcolor[rgb]{0.73,0.40,0.53}{##1}}}
\expandafter\def\csname PY@tok@se\endcsname{\let\PY@bf=\textbf\def\PY@tc##1{\textcolor[rgb]{0.73,0.40,0.13}{##1}}}
\expandafter\def\csname PY@tok@sr\endcsname{\def\PY@tc##1{\textcolor[rgb]{0.73,0.40,0.53}{##1}}}
\expandafter\def\csname PY@tok@ss\endcsname{\def\PY@tc##1{\textcolor[rgb]{0.10,0.09,0.49}{##1}}}
\expandafter\def\csname PY@tok@sx\endcsname{\def\PY@tc##1{\textcolor[rgb]{0.00,0.50,0.00}{##1}}}
\expandafter\def\csname PY@tok@m\endcsname{\def\PY@tc##1{\textcolor[rgb]{0.40,0.40,0.40}{##1}}}
\expandafter\def\csname PY@tok@gh\endcsname{\let\PY@bf=\textbf\def\PY@tc##1{\textcolor[rgb]{0.00,0.00,0.50}{##1}}}
\expandafter\def\csname PY@tok@gu\endcsname{\let\PY@bf=\textbf\def\PY@tc##1{\textcolor[rgb]{0.50,0.00,0.50}{##1}}}
\expandafter\def\csname PY@tok@gd\endcsname{\def\PY@tc##1{\textcolor[rgb]{0.63,0.00,0.00}{##1}}}
\expandafter\def\csname PY@tok@gi\endcsname{\def\PY@tc##1{\textcolor[rgb]{0.00,0.63,0.00}{##1}}}
\expandafter\def\csname PY@tok@gr\endcsname{\def\PY@tc##1{\textcolor[rgb]{1.00,0.00,0.00}{##1}}}
\expandafter\def\csname PY@tok@ge\endcsname{\let\PY@it=\textit}
\expandafter\def\csname PY@tok@gs\endcsname{\let\PY@bf=\textbf}
\expandafter\def\csname PY@tok@gp\endcsname{\let\PY@bf=\textbf\def\PY@tc##1{\textcolor[rgb]{0.00,0.00,0.50}{##1}}}
\expandafter\def\csname PY@tok@go\endcsname{\def\PY@tc##1{\textcolor[rgb]{0.53,0.53,0.53}{##1}}}
\expandafter\def\csname PY@tok@gt\endcsname{\def\PY@tc##1{\textcolor[rgb]{0.00,0.27,0.87}{##1}}}
\expandafter\def\csname PY@tok@err\endcsname{\def\PY@bc##1{\setlength{\fboxsep}{0pt}\fcolorbox[rgb]{1.00,0.00,0.00}{1,1,1}{\strut ##1}}}
\expandafter\def\csname PY@tok@kc\endcsname{\let\PY@bf=\textbf\def\PY@tc##1{\textcolor[rgb]{0.00,0.50,0.00}{##1}}}
\expandafter\def\csname PY@tok@kd\endcsname{\let\PY@bf=\textbf\def\PY@tc##1{\textcolor[rgb]{0.00,0.50,0.00}{##1}}}
\expandafter\def\csname PY@tok@kn\endcsname{\let\PY@bf=\textbf\def\PY@tc##1{\textcolor[rgb]{0.00,0.50,0.00}{##1}}}
\expandafter\def\csname PY@tok@kr\endcsname{\let\PY@bf=\textbf\def\PY@tc##1{\textcolor[rgb]{0.00,0.50,0.00}{##1}}}
\expandafter\def\csname PY@tok@bp\endcsname{\def\PY@tc##1{\textcolor[rgb]{0.00,0.50,0.00}{##1}}}
\expandafter\def\csname PY@tok@fm\endcsname{\def\PY@tc##1{\textcolor[rgb]{0.00,0.00,1.00}{##1}}}
\expandafter\def\csname PY@tok@vc\endcsname{\def\PY@tc##1{\textcolor[rgb]{0.10,0.09,0.49}{##1}}}
\expandafter\def\csname PY@tok@vg\endcsname{\def\PY@tc##1{\textcolor[rgb]{0.10,0.09,0.49}{##1}}}
\expandafter\def\csname PY@tok@vi\endcsname{\def\PY@tc##1{\textcolor[rgb]{0.10,0.09,0.49}{##1}}}
\expandafter\def\csname PY@tok@vm\endcsname{\def\PY@tc##1{\textcolor[rgb]{0.10,0.09,0.49}{##1}}}
\expandafter\def\csname PY@tok@sa\endcsname{\def\PY@tc##1{\textcolor[rgb]{0.73,0.13,0.13}{##1}}}
\expandafter\def\csname PY@tok@sb\endcsname{\def\PY@tc##1{\textcolor[rgb]{0.73,0.13,0.13}{##1}}}
\expandafter\def\csname PY@tok@sc\endcsname{\def\PY@tc##1{\textcolor[rgb]{0.73,0.13,0.13}{##1}}}
\expandafter\def\csname PY@tok@dl\endcsname{\def\PY@tc##1{\textcolor[rgb]{0.73,0.13,0.13}{##1}}}
\expandafter\def\csname PY@tok@s2\endcsname{\def\PY@tc##1{\textcolor[rgb]{0.73,0.13,0.13}{##1}}}
\expandafter\def\csname PY@tok@sh\endcsname{\def\PY@tc##1{\textcolor[rgb]{0.73,0.13,0.13}{##1}}}
\expandafter\def\csname PY@tok@s1\endcsname{\def\PY@tc##1{\textcolor[rgb]{0.73,0.13,0.13}{##1}}}
\expandafter\def\csname PY@tok@mb\endcsname{\def\PY@tc##1{\textcolor[rgb]{0.40,0.40,0.40}{##1}}}
\expandafter\def\csname PY@tok@mf\endcsname{\def\PY@tc##1{\textcolor[rgb]{0.40,0.40,0.40}{##1}}}
\expandafter\def\csname PY@tok@mh\endcsname{\def\PY@tc##1{\textcolor[rgb]{0.40,0.40,0.40}{##1}}}
\expandafter\def\csname PY@tok@mi\endcsname{\def\PY@tc##1{\textcolor[rgb]{0.40,0.40,0.40}{##1}}}
\expandafter\def\csname PY@tok@il\endcsname{\def\PY@tc##1{\textcolor[rgb]{0.40,0.40,0.40}{##1}}}
\expandafter\def\csname PY@tok@mo\endcsname{\def\PY@tc##1{\textcolor[rgb]{0.40,0.40,0.40}{##1}}}
\expandafter\def\csname PY@tok@ch\endcsname{\let\PY@it=\textit\def\PY@tc##1{\textcolor[rgb]{0.25,0.50,0.50}{##1}}}
\expandafter\def\csname PY@tok@cm\endcsname{\let\PY@it=\textit\def\PY@tc##1{\textcolor[rgb]{0.25,0.50,0.50}{##1}}}
\expandafter\def\csname PY@tok@cpf\endcsname{\let\PY@it=\textit\def\PY@tc##1{\textcolor[rgb]{0.25,0.50,0.50}{##1}}}
\expandafter\def\csname PY@tok@c1\endcsname{\let\PY@it=\textit\def\PY@tc##1{\textcolor[rgb]{0.25,0.50,0.50}{##1}}}
\expandafter\def\csname PY@tok@cs\endcsname{\let\PY@it=\textit\def\PY@tc##1{\textcolor[rgb]{0.25,0.50,0.50}{##1}}}

\def\PYZbs{\char`\\}
\def\PYZus{\char`\_}
\def\PYZob{\char`\{}
\def\PYZcb{\char`\}}
\def\PYZca{\char`\^}
\def\PYZam{\char`\&}
\def\PYZlt{\char`\<}
\def\PYZgt{\char`\>}
\def\PYZsh{\char`\#}
\def\PYZpc{\char`\%}
\def\PYZdl{\char`\$}
\def\PYZhy{\char`\-}
\def\PYZsq{\char`\'}
\def\PYZdq{\char`\"}
\def\PYZti{\char`\~}
% for compatibility with earlier versions
\def\PYZat{@}
\def\PYZlb{[}
\def\PYZrb{]}
\makeatother


    % For linebreaks inside Verbatim environment from package fancyvrb. 
    \makeatletter
        \newbox\Wrappedcontinuationbox 
        \newbox\Wrappedvisiblespacebox 
        \newcommand*\Wrappedvisiblespace {\textcolor{red}{\textvisiblespace}} 
        \newcommand*\Wrappedcontinuationsymbol {\textcolor{red}{\llap{\tiny$\m@th\hookrightarrow$}}} 
        \newcommand*\Wrappedcontinuationindent {3ex } 
        \newcommand*\Wrappedafterbreak {\kern\Wrappedcontinuationindent\copy\Wrappedcontinuationbox} 
        % Take advantage of the already applied Pygments mark-up to insert 
        % potential linebreaks for TeX processing. 
        %        {, <, #, %, $, ' and ": go to next line. 
        %        _, }, ^, &, >, - and ~: stay at end of broken line. 
        % Use of \textquotesingle for straight quote. 
        \newcommand*\Wrappedbreaksatspecials {% 
            \def\PYGZus{\discretionary{\char`\_}{\Wrappedafterbreak}{\char`\_}}% 
            \def\PYGZob{\discretionary{}{\Wrappedafterbreak\char`\{}{\char`\{}}% 
            \def\PYGZcb{\discretionary{\char`\}}{\Wrappedafterbreak}{\char`\}}}% 
            \def\PYGZca{\discretionary{\char`\^}{\Wrappedafterbreak}{\char`\^}}% 
            \def\PYGZam{\discretionary{\char`\&}{\Wrappedafterbreak}{\char`\&}}% 
            \def\PYGZlt{\discretionary{}{\Wrappedafterbreak\char`\<}{\char`\<}}% 
            \def\PYGZgt{\discretionary{\char`\>}{\Wrappedafterbreak}{\char`\>}}% 
            \def\PYGZsh{\discretionary{}{\Wrappedafterbreak\char`\#}{\char`\#}}% 
            \def\PYGZpc{\discretionary{}{\Wrappedafterbreak\char`\%}{\char`\%}}% 
            \def\PYGZdl{\discretionary{}{\Wrappedafterbreak\char`\$}{\char`\$}}% 
            \def\PYGZhy{\discretionary{\char`\-}{\Wrappedafterbreak}{\char`\-}}% 
            \def\PYGZsq{\discretionary{}{\Wrappedafterbreak\textquotesingle}{\textquotesingle}}% 
            \def\PYGZdq{\discretionary{}{\Wrappedafterbreak\char`\"}{\char`\"}}% 
            \def\PYGZti{\discretionary{\char`\~}{\Wrappedafterbreak}{\char`\~}}% 
        } 
        % Some characters . , ; ? ! / are not pygmentized. 
        % This macro makes them "active" and they will insert potential linebreaks 
        \newcommand*\Wrappedbreaksatpunct {% 
            \lccode`\~`\.\lowercase{\def~}{\discretionary{\hbox{\char`\.}}{\Wrappedafterbreak}{\hbox{\char`\.}}}% 
            \lccode`\~`\,\lowercase{\def~}{\discretionary{\hbox{\char`\,}}{\Wrappedafterbreak}{\hbox{\char`\,}}}% 
            \lccode`\~`\;\lowercase{\def~}{\discretionary{\hbox{\char`\;}}{\Wrappedafterbreak}{\hbox{\char`\;}}}% 
            \lccode`\~`\:\lowercase{\def~}{\discretionary{\hbox{\char`\:}}{\Wrappedafterbreak}{\hbox{\char`\:}}}% 
            \lccode`\~`\?\lowercase{\def~}{\discretionary{\hbox{\char`\?}}{\Wrappedafterbreak}{\hbox{\char`\?}}}% 
            \lccode`\~`\!\lowercase{\def~}{\discretionary{\hbox{\char`\!}}{\Wrappedafterbreak}{\hbox{\char`\!}}}% 
            \lccode`\~`\/\lowercase{\def~}{\discretionary{\hbox{\char`\/}}{\Wrappedafterbreak}{\hbox{\char`\/}}}% 
            \catcode`\.\active
            \catcode`\,\active 
            \catcode`\;\active
            \catcode`\:\active
            \catcode`\?\active
            \catcode`\!\active
            \catcode`\/\active 
            \lccode`\~`\~ 	
        }
    \makeatother

    \let\OriginalVerbatim=\Verbatim
    \makeatletter
    \renewcommand{\Verbatim}[1][1]{%
        %\parskip\z@skip
        \sbox\Wrappedcontinuationbox {\Wrappedcontinuationsymbol}%
        \sbox\Wrappedvisiblespacebox {\FV@SetupFont\Wrappedvisiblespace}%
        \def\FancyVerbFormatLine ##1{\hsize\linewidth
            \vtop{\raggedright\hyphenpenalty\z@\exhyphenpenalty\z@
                \doublehyphendemerits\z@\finalhyphendemerits\z@
                \strut ##1\strut}%
        }%
        % If the linebreak is at a space, the latter will be displayed as visible
        % space at end of first line, and a continuation symbol starts next line.
        % Stretch/shrink are however usually zero for typewriter font.
        \def\FV@Space {%
            \nobreak\hskip\z@ plus\fontdimen3\font minus\fontdimen4\font
            \discretionary{\copy\Wrappedvisiblespacebox}{\Wrappedafterbreak}
            {\kern\fontdimen2\font}%
        }%
        
        % Allow breaks at special characters using \PYG... macros.
        \Wrappedbreaksatspecials
        % Breaks at punctuation characters . , ; ? ! and / need catcode=\active 	
        \OriginalVerbatim[#1,codes*=\Wrappedbreaksatpunct]%
    }
    \makeatother

    % Exact colors from NB
    \definecolor{incolor}{HTML}{303F9F}
    \definecolor{outcolor}{HTML}{D84315}
    \definecolor{cellborder}{HTML}{CFCFCF}
    \definecolor{cellbackground}{HTML}{F7F7F7}
    
    % prompt
    \makeatletter
    \newcommand{\boxspacing}{\kern\kvtcb@left@rule\kern\kvtcb@boxsep}
    \makeatother
    \newcommand{\prompt}[4]{
        \ttfamily\llap{{\color{#2}[#3]:\hspace{3pt}#4}}\vspace{-\baselineskip}
    }
    

    
    % Prevent overflowing lines due to hard-to-break entities
    \sloppy 
    % Setup hyperref package
    \hypersetup{
      breaklinks=true,  % so long urls are correctly broken across lines
      colorlinks=true,
      urlcolor=urlcolor,
      linkcolor=linkcolor,
      citecolor=citecolor,
      }
    % Slightly bigger margins than the latex defaults
    
    \geometry{verbose,tmargin=1in,bmargin=1in,lmargin=1in,rmargin=1in}
    
    

\begin{document}
    
    \maketitle
    
    

    
    \hypertarget{benefits-of-regex}{%
\subsubsection{Benefits of Regex:}\label{benefits-of-regex}}

\begin{itemize}
\tightlist
\item
  complex operations with string data can be written a lot quicker,
  which will save you time.
\item
  Regular expressions are often faster to execute than their manual
  equivalents.
\item
  Regular expressions are supported in almost every modern programming
  language, as well as other places like command line utilities and
  databases. Understanding regular expressions gives you a powerful tool
  that you can use wherever you work with data.
\end{itemize}

    \hypertarget{dont-memorize-syntax}{%
\subsubsection{Don't memorize syntax !!}\label{dont-memorize-syntax}}

\begin{itemize}
\item
  One thing to keep in mind before we start: don't expect to remember
  all of the regular expression syntax. The most important thing is to
  understand the core principles, what is possible, and where to look up
  the details. This will mean you can quickly jog your memory whenever
  you need regular expressions.
\item
  As long as you are able to write and understand regular expressions
  with the help of documentation and/or other reference guides, you have
  all the skills you need to excel.
\end{itemize}

    \begin{itemize}
\tightlist
\item
  When working with regular expressions, we use the term pattern to
  describe a regular expression that we've written. If the pattern is
  found within the string we're searching, we say that it has matched.
\item
  We previously used regular expressions with pandas, but Python also
  has a built-in module for regular expressions: The re module. This
  module contains a number of different functions and classes for
  working with regular expressions. One of the most useful functions
  from the re module is the re.search() function, which takes two
  required arguments:

  \begin{itemize}
  \tightlist
  \item
    The regex pattern
  \item
    The string we want to search that pattern for
  \end{itemize}
\item
  The re.search() function will return a Match object if the pattern is
  found anywhere within the string. If the pattern is not found,
  re.search() returns None
\item
  The first of these we'll learn is called a set. A set allows us to
  specify two or more characters that can match in a single character's
  position.
\end{itemize}

\begin{figure}
\centering
\includegraphics{attachment:set_syntax_breakdown.svg}
\caption{set\_syntax\_breakdown.svg}
\end{figure}

\begin{itemize}
\tightlist
\item
  We've learned that we should avoid using loops in pandas, and that
  vectorized methods are often faster and require less code.we learned
  that the Series.str.contains() method can be used to test whether a
  Series of strings match a particular regex pattern.
\end{itemize}

\begin{verbatim}
eg_list = ["Julie's favorite color is green.",
           "Keli's favorite color is Blue.",
           "Craig's favorite colors are blue and red."]

eg_series = pd.Series(eg_list)

pattern = "[Bb]lue"

pattern_contained = eg_series.str.contains(pattern)
print(pattern_contained)
0    False
1     True
2     True
dtype: bool
\end{verbatim}

\begin{itemize}
\tightlist
\item
  One of the neat things about boolean masks is that you can use the
  Series.sum() method to sum all the values in the boolean mask, with
  each True value counting as 1, and each False as 0. This means that we
  can easily count the number of values in the original series that
  matched our pattern:
\end{itemize}

\begin{verbatim}
pattern_count = pattern_contained.sum()
\end{verbatim}

\hypertarget{set-and-quantifiers}{%
\subsubsection{Set {[} {]} and Quantifiers}\label{set-and-quantifiers}}

\begin{itemize}
\tightlist
\item
  we learned that we could use braces (\{\}) to specify that a character
  repeats in our regular expression. For instance, if we wanted to write
  a pattern that matches the numbers in text from 1000 to 2999 we could
  write the regular expression below:
\end{itemize}

\begin{figure}
\centering
\includegraphics{attachment:quantifier_example.svg}
\caption{quantifier\_example.svg}
\end{figure}

\begin{itemize}
\item
  The name for this type of regular expression syntax is called a
  quantifier. Quantifiers specify how many of the previous character our
  pattern requires, which can help us when we want to match substrings
  of specific lengths. As an example, we might want to match both e-mail
  and email. To do this, we would want to specify to match - either zero
  or one times.
\item
  The specific type of quantifier we saw above is called a numeric
  quantifier. Here are the different types of numeric quantifiers we can
  use:
\end{itemize}

\begin{figure}
\centering
\includegraphics{attachment:quantifiers_other.svg}
\caption{quantifiers\_other.svg}
\end{figure}

\begin{itemize}
\tightlist
\item
  quantifiers with other stuff:
\end{itemize}

\begin{figure}
\centering
\includegraphics{attachment:character_classes_v2_1.svg}
\caption{character\_classes\_v2\_1.svg}
\end{figure}

\begin{itemize}
\tightlist
\item
  Just like with quantifiers, there are some other common character
  classes which we'll use a lot.
  \includegraphics{attachment:character_classes_v2_2.svg}
\end{itemize}

-Let's quickly recap the concepts we learned in this screen:

\begin{verbatim}
-We can use a backslash to escape characters that have special meaning in regular expressions (e.g. \ will match an open bracket character).
-Character classes let us match certain groups of characters (e.g. \w will match any word character).
-Character classes can be combined with quantifiers when we want to match different numbers of characters.
\end{verbatim}

\begin{itemize}
\tightlist
\item
  since set {[}{]} has a special meaning but say you want to find a
  pattern where the titles had {[}{]} in them you can use escape
  characters to match those {[}{]}.
\end{itemize}

\begin{verbatim}
print(r'hello\b')
hello
\end{verbatim}

\begin{itemize}
\tightlist
\item
  \b has a special meaning(backsplash) But by using raw string `r'
  infront these special character interpreatations will be turned
  off.It's strongly recommend using raw strings for every regex you
  write, rather than remember which sequences are escape sequences and
  using raw strings selectively. That way, you'll never encounter a
  situation where you forget or overlook something which causes your
  regex to break.
\end{itemize}

\hypertarget{capture-groups}{%
\subsubsection{Capture Groups ``()''}\label{capture-groups}}

\begin{itemize}
\tightlist
\item
  What if we wanted to find out what the text of matched part of the
  string? In order to do this, we'll need to use capture groups. Capture
  groups allow us to specify one or more groups within our match that we
  can access separately. In this mission, we'll learn how to use one
  capture group per regular expression, but in the next mission we'll
  learn some more complex capture group patterns.
\end{itemize}

\begin{verbatim}
tag_5 = tag_titles.head()

67      Analysis of 114 propaganda sources from ISIS, Jabhat al-Nusra, al-Qaeda [pdf]
101                                Munich Gunman Got Weapon from the Darknet [German]
160                                      File indexing and searching for Plan 9 [pdf]
163    Attack on Kunduz Trauma Centre, Afghanistan  Initial MSF Internal Review [pdf]
196                                            [Beta] Speedtest.net  HTML5 Speed Test
Name: title, dtype: object

pattern = r"(\[\w+\])"
tag_5_matches = tag_5.str.extract(pattern)
print(tag_5_matches)
67        [pdf]
101    [German]
160       [pdf]
163       [pdf]
196      [Beta]
Name: title, dtype: object

pattern = r"\[(\w+)\]"
tag_5_matches = tag_5.str.extract(pattern)
print(tag_5_matches)

67        pdf
101    German
160       pdf
163       pdf
196      Beta
Name: title, dtype: object
\end{verbatim}

\begin{itemize}
\tightlist
\item
  \textbf{Negative character classes}
\end{itemize}

\begin{figure}
\centering
\includegraphics{attachment:negative_character_classes.svg}
\caption{negative\_character\_classes.svg}
\end{figure}

\begin{verbatim}
PS : when using negative classes it fails for the cases where the pattern occurs at the end of the string.
for example: pattern = r'([Jj]ava[^Ss])'it helps in avoiding to find strings that contains Javascript and helps in finding strings containing java but it fails if java occurs at the end of the string since there is no string left to do the matching it fails to match.
\end{verbatim}

\begin{itemize}
\item
  \textbf{word boundary anchor}: In the above case if we use
  r'\b[Jj]ava\b' we can find only strings containing word Java not java
  script and it will match even if the word java occurs at the end of
  the sentence.
\item
  \textbf{Start and End anchor }:
\end{itemize}

\begin{figure}
\centering
\includegraphics{attachment:positional_anchors.svg}
\caption{positional\_anchors.svg}
\end{figure}

\begin{verbatim}
Note that the ^ character is used both as a beginning anchor and to indicate a negative set, depending on whether the character preceding it is a [ or not.
e.g [^sS] not containing S or s.
    ^s starts with s
\end{verbatim}

\hypertarget{ignore-capitilization-cases}{%
\subsubsection{Ignore Capitilization
cases}\label{ignore-capitilization-cases}}

\begin{itemize}
\item
  Up until now, we've been using sets like {[}Pp{]} to match different
  capitalizations in our regular expressions. This strategy works well
  when there is only one character that has capitalization, but becomes
  cumbersome when we need to cater for multiple instances.

  Within the titles, there are many different formatting styles used to
  represent the word ``email.'' Here is a list of the variations:

\begin{verbatim}
email
Email
e Mail
e mail
E-mail
e-mail
eMail
E-Mail
EMAIL
\end{verbatim}

\begin{verbatim}
To write a regular expression for this, we would need to use a set for all five letters in email, which would make our regular expression very hard to read.
\end{verbatim}

  Instead, we can use flags to specify that our regular expression
  should ignore case.

  Both re.search() and the pandas regular expression methods accept an
  optional flags argument. This argument accepts one or more flags,
  which are special variables in the re module that modify the behavior
  of the regex interpreter.

  Usage:
  \texttt{import\ re\ \ \ email\_tests.str.contains(r"email",flags=re.I)}
\end{itemize}

    \hypertarget{advanced-regular-expressions}{%
\subsection{Advanced Regular
Expressions}\label{advanced-regular-expressions}}

    \begin{itemize}
\item
  As we learned in the previous mission, to extract those mentions, we
  need to do two things:

  \begin{enumerate}
  \def\labelenumi{\arabic{enumi}.}
  \tightlist
  \item
    Use the Series.str.extract() method.
  \item
    Use a regex capture group.
  \end{enumerate}
\end{itemize}

\hypertarget{look-arounds}{%
\subsubsection{Look arounds}\label{look-arounds}}

\begin{itemize}
\tightlist
\item
  Lookarounds let us define a character or sequence of characters that
  either must or must not come before or after our regex match. There
  are four types of lookarounds:
\end{itemize}

\begin{figure}
\centering
\includegraphics{attachment:lookarounds.svg}
\caption{lookarounds.svg}
\end{figure}

\begin{itemize}
\item
  These tips can help you remember the syntax for lookarounds:

  \begin{enumerate}
  \def\labelenumi{\arabic{enumi}.}
  \tightlist
  \item
    Inside the parentheses, the first character of a lookaround is
    always ?.
  \item
    If the lookaround is a lookbehind, the next character will be
    \textless{}, which you can think of as an arrow head pointing behind
    the match.
  \item
    The next character indicates whether the is lookaround is positive
    (=) or negative (!).
  \end{enumerate}
\end{itemize}

\hypertarget{back-references-1oo-bb-aa}{%
\subsubsection{\texorpdfstring{Back references
(\w)\textbackslash{}1--oo-bb-aa}{Back references ()\textbackslash{}1--oo-bb-aa}}\label{back-references-1oo-bb-aa}}

\begin{itemize}
\item
  Let's say we wanted to identify strings that had words with double
  letters, like the ``ee'' in ``feed.'' Because we don't know ahead of
  time what letters might be repeated, we need a way to specify a
  capture group and then to repeat it. We can do this with
  backreferences.

\begin{verbatim}
test_cases = [
            "I'm going to read a book.",
            "Green is my favorite color.",
            "My name is Aaron.",
            "No doubles here.",
            "I have a pet eel."
           ]

for tc in test_cases:
  print(re.search(r"(\w)\1", tc))


  _sre.SRE_Match object; span=(21, 23), match='oo'
  _sre.SRE_Match object; span=(2, 4), match='ee'
  None
  None
  _sre.SRE_Match object; span=(13, 15), match='ee'
\end{verbatim}
\end{itemize}

\hypertarget{string-replace-or-regex-sub-substitute}{%
\subsubsection{string replace or regex sub
(substitute)}\label{string-replace-or-regex-sub-substitute}}

\begin{itemize}
\tightlist
\item
  When we learned to work with basic string methods, we used the
  str.replace() method to replace simple substrings. We can achieve the
  same with regular expressions using the re.sub() function. The basic
  syntax for re.sub() is:
\end{itemize}

\begin{verbatim}
string = "aBcDEfGHIj"

print(re.sub(r"[A-Z]", "-", string))

a-c--f---j
\end{verbatim}

\hypertarget{example-of-using-capture-groups-to-capture-mutiple-parts-of-an-url}{%
\subsubsection{Example of using capture groups to capture mutiple parts
of an
URL}\label{example-of-using-capture-groups-to-capture-mutiple-parts-of-an-url}}

\begin{itemize}
\item
  we'll extract each of the three component parts of the URLs:

  \begin{enumerate}
  \def\labelenumi{\arabic{enumi}.}
  \tightlist
  \item
    Protocol
  \item
    Domain
  \item
    Page path
  \end{enumerate}
\end{itemize}

\begin{figure}
\centering
\includegraphics{attachment:url_examples_2.svg}
\caption{url\_examples\_2.svg}
\end{figure}

\begin{itemize}
\item
  In order to do this, we'll create a regular expression with multiple
  capture groups. Multiple capture groups in regular expressions are
  defined the same way as single capture groups --- using pairs of
  parentheses.
\item
  if you have some data that looks like this below:

\begin{verbatim}
  0     8/4/2016 11:52
  1    1/26/2016 19:30
  2    6/23/2016 22:20
  3     6/17/2016 0:01
  4     9/30/2015 4:12
  Name: created_at, dtype: object
\end{verbatim}
\item
  We'll use capture groups to extract these dates and times into two
  columns:
\end{itemize}

\begin{longtable}[]{@{}ll@{}}
\toprule
Date & Time\tabularnewline
\midrule
\endhead
8/4/2016 & 11:52\tabularnewline
1/26/2016 & 19:30\tabularnewline
6/23/2016 & 22:20\tabularnewline
6/17/2016 & 0:01\tabularnewline
9/30/2015 & 4:12\tabularnewline
\bottomrule
\end{longtable}

\begin{itemize}
\tightlist
\item
  In order to do this we can write the following regular expression:
\end{itemize}

\begin{figure}
\centering
\includegraphics{attachment:multiple_capture_groups.svg}
\caption{multiple\_capture\_groups.svg}
\end{figure}

\begin{itemize}
\item
  Similarly I have these test URL's as shown below:

\begin{verbatim}
test_urls = pd.Series([
 'https://www.amazon.com/Technology-Ventures-Enterprise-Thomas-Byers/dp/0073523429',
 'http://www.interactivedynamicvideo.com/',
 'http://www.nytimes.com/2007/11/07/movies/07stein.html?_r=0',
 'http://evonomics.com/advertising-cannot-maintain-internet-heres-solution/',
 'HTTPS://github.com/keppel/pinn',
 'Http://phys.org/news/2015-09-scale-solar-youve.html',
 'https://iot.seeed.cc',
 'http://www.bfilipek.com/2016/04/custom-deleters-for-c-smart-pointers.html',
 'http://beta.crowdfireapp.com/?beta=agnipath',
 'https://www.valid.ly?param'
])

pattern  = r '(.+)://([\w+.]*)[/]?(.*)'

url_parts = test_urls.str.extarct(pattern, flags = re.I)


\end{verbatim}
\item
  The resulting dataframe looks like this:

  \begin{longtable}[]{@{}lll@{}}
  \toprule
  0 & 1 & 2\tabularnewline
  \midrule
  \endhead
  https & www.amazon.com &
  Technology-Ventures-Enterprise-Thomas-Byers/dp\ldots{}\tabularnewline
  http & www.interactivedynamicvideo.com &\tabularnewline
  http & www.nytimes.com &
  2007/11/07/movies/07stein.html?\_r=0\tabularnewline
  http & evonomics.com &
  advertising-cannot-maintain-internet-heres-sol\ldots{}\tabularnewline
  HTTPS & github.com & keppel/pinn\tabularnewline
  \bottomrule
  \end{longtable}
\item
  if we want names for our capture groups we can use the following
  syntax
\end{itemize}

\begin{figure}
\centering
\includegraphics{attachment:named_capture_groups.svg}
\caption{named\_capture\_groups.svg}
\end{figure}

\hypertarget{thank}{%
\subsubsection{Thank}\label{thank}}

    \begin{tcolorbox}[breakable, size=fbox, boxrule=1pt, pad at break*=1mm,colback=cellbackground, colframe=cellborder]
\prompt{In}{incolor}{ }{\boxspacing}
\begin{Verbatim}[commandchars=\\\{\}]

\end{Verbatim}
\end{tcolorbox}


    % Add a bibliography block to the postdoc
    
    
    
\end{document}
